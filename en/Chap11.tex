% LaTeX/AMS-LaTeX

\documentclass[a4paper,11pt]{book}

%%% remove comment delimiter ('%') and specify encoding parameter if required,
%%% see TeX documentation for additional info (cp1252-Western,cp1251-Cyrillic)
%\usepackage[cp1252]{inputenc}

%%% remove comment delimiter ('%') and select language if required
%\usepackage[english,spanish]{babel}

\usepackage{amssymb}
\usepackage{amsmath}
\usepackage[dvips]{graphicx}
%%% remove comment delimiter ('%') and specify parameters if required
%\usepackage[dvips]{graphics}

\begin{document}

%%% remove comment delimiter ('%') and select language if required
%\selectlanguage{spanish} 

\noindent 

\noindent 

\noindent 

\noindent 

\noindent 

\noindent 

\noindent 

\noindent 

\noindent 

\noindent 

\noindent 

\noindent 

\noindent 

\noindent 

\noindent 

\noindent Object-Oriented Perl

\noindent 

\noindent 

\noindent 

\noindent 

\noindent As we've mentioned before, there are several schools of thought in programming. One in particular has gained a lot of popularity over the past five or ten years -- it's called \textbf{object-oriented programming}, or \textbf{OOP }for short. The type of programming we've been doing so far has been based around tasks --

\noindent splitting projects up into smaller and smaller tasks, using subroutines to define a single task, and so on. This is called \textbf{procedural programming}, because the focus is on the procedures involved in getting the job done. In object-oriented programming, on the other hand, the focus is on the data. Chunks of data called \textbf{objects }can have various properties and can interact with each other in various ways.

\noindent 

\noindent In this chapter, we'll learn how to start thinking in object-oriented (OO) terms. OO involves a lot of jargon, so the first thing we'll do is look at all the new terms associated with OO and what they mean to

\noindent a Perl programmer. After that, we'll see how to approach a problem using this style of programming. We'll use some CPAN modules that involve objects, and we'll also construct some object-oriented modules of our own.

\noindent 

\noindent Finally, we'll examine ties, which offer a way to hide the workings of a module behind an ordinary looking variable.

\noindent 

\noindent 

\noindent Working with Objects

\noindent Procedural programming deals with tasks; the basic unit of operation is the subroutine, which describes how a task is carried out. It's also concerned with breaking tasks down into smaller and smaller stages until they become easy to describe to the computer.

\noindent 

\noindent Object-oriented programming, on the other hand, is more concerned with groups of actions and interactions between data. Here, the basic unit of operation is the object (a chunk of data). Attached to that chunk of data is a set of controls that the user can use to interact with it.

\noindent 

\noindent For instance, if you're writing a calendar program for keeping track of appointments, there aren't any specific tasks that are fundamental to the usefulness of the calendar. What is fundamental is the chunk of data that represents your diary -- so you'd start off by creating a diary object. There are certain things

\noindent you might want that diary object to do: print out a calendar, tell you today's appointments, and so on. You could also ask that diary object to create an appointment. However, this would require creating a

\noindent different sort of chunk of data, to represent an appointment. All the data for a given appointment would be stored in an appointment object, which would then be attached to the diary:

\noindent  

\noindent  

\noindent  

\noindent  

\noindent 

\noindent 

\noindent 

\noindent Methods

\noindent 

\noindent 

\noindent 

\noindent Attributes

\noindent 

\noindent 

\noindent 

\noindent 

\noindent 

\noindent 

\noindent 

\noindent Appointment Object

\noindent 

\noindent Private Data

\noindent Appointment Object

\noindent 

\noindent Private Data

\noindent 

\noindent Appointment

\noindent Class

\noindent 

\noindent 

\noindent 

\noindent 

\noindent 

\noindent Turning Tasks into OO Programs

\noindent 

\noindent So how do you decide whether or not you should be using a procedural or an OO style in your programs? Here are five guidelines to help you decide.

\noindent 

\noindent \textit{Are your subroutines tasks?}

\noindent If your program naturally involves a series of unconnected tasks, you probably want to be using a procedural style. If your application is \textbf{data-driven}, then you're dealing primarily with data structures rather than tasks, so consider using an OO style instead.

\noindent 

\noindent \textit{Do you need persistence?}

\noindent After your task is completed, do you need somewhere to store data that you want to receive next time

\noindent you process that data? If so, you may find it easier to use an OO interface. If each call to a subroutine is completely independent of the others, you can use a procedural interface.

\noindent 

\noindent For instance, if you're producing a cross-reference table, your cross-reference subroutine will need to know whether or not the thing it's processing has turned up before or not. Since an object packages up everything we know about a piece of data, it's easy to deal with that directly.

\noindent 

\noindent \textit{Do you need sessions?}

\noindent Do you want to process several different chunks of data with the same subroutines? For instance, if you have two different 'sessions' that signify database connections or network connections, you will find it much, much easier to package up each session into an object.

\noindent 

\noindent \textit{Do you need OO?}

\noindent Object-oriented programs run slightly slower than equally well-written procedural programs that do the same job, because packaging things into objects and passing objects around is expensive, both in

\noindent terms of time spent and resources used. If you can get away without using object orientation, you

\noindent probably should.

\noindent 

\noindent 

\noindent \textit{Do you want the user to be unaware of the object?}

\noindent If you need to store an object, but want to hide away any processing you do with it behind an ordinary variable, you need to use a part of object-oriented programming called 'tying'. For example, as we'll see

\noindent in Chapter 13, you can hide a database object behind an ordinary hash, so that when you look something up in the hash, the object looks it up in the database on disk. When something is stored in

\noindent the hash, the object writes the data to the disk. The end-user is not necessarily aware that he (or she) is dealing with an object, but some special things happen when s/he accesses the hash.

\noindent 

\noindent \textit{Are you still unsure?}

\noindent Unless you know you need an OO model, it's probably better to use a procedural model to help maintenance and readability. If you're still unsure, go with an ordinary procedural model.

\noindent 

\noindent Improving Your Vocabulary

\noindent 

\noindent 'Object-oriented programming' wouldn't be a good 'buzz-phrase' if it didn't use a lot of familiar words in unfamiliar contexts. Before we go any further, let's investigate the jargon terms that we'll need in order

\noindent to understand Perl OOP.

\noindent 

\noindent The first thing to note is that OOP is a concept, rather than a standard. There are a few things that OO languages should do, a lot they can do, but nothing that they absolutely \textit{have }to do. Other languages may implement more or less of these ideas than Perl does and may well do so in a completely different way. We'll explain here the terms that are most commonly used by object-oriented programmers.

\noindent 

\noindent \textit{Objects}

\noindent What is an object, anyway? I mentioned briefly above that an object is a chunk of data -- but that's not all. To be honest, an object can be anything -- it really depends on what your application is. For

\noindent instance, if you're writing a contact management database, a single contact might be an object. If you're communicating with a remote computer via FTP, you could make each connection to the remote server

\noindent an object.

\noindent 

\noindent An object can always be described in terms of two things:

\noindent 

\noindent ? what it can do

\noindent 

\noindent ? what we know about it

\noindent 

\noindent With a 'contact  record' object,  we'd  probably  know  the  contact's  name,  date  of  birth,  address,  and  so

\noindent on.  These are the object's \textbf{attributes}.  We might  also  be  able  to  ask it  to  do  certain  things:  print  an address  label for this contact,  work  out  how  old  they  are,  or  send  them  an  email.  These  are the

\noindent object's \textbf{methods}.

\noindent 

\noindent In Perl, what we see as an object is simply a reference. In fact, you can convert any ordinary reference \textit{into }an object simply by using the bless operator. We'll see later on how that happens. Typically, however, objects are represented as references to a hash, and that's the model we'll use in this chapter.

\noindent 

\noindent 

\noindent \textit{Attributes}

\noindent As we've just seen, an attribute is something we know about an object. A contact database object will possess attributes such as date of birth, address, and name. An FTP session will possess attributes such

\noindent as the name of the remote server we're connected to, the current directory, and so on. Two contacts will have different values for their name attribute (unless we have duplicates in the database), but they will

\noindent both have the name attribute.

\noindent 

\noindent If we're using a reference to a hash, it's natural to have the attributes as hash entries. Our person object then becomes a blessed version of the following:

\noindent 

\noindent my \$person = \{

\noindent surname =$>$ "Galilei", forename =$>$ "Galileo",

\noindent address =$>$ "9.81 Pisa Apts.", occupation =$>$ "bombadier"

\noindent \};

\noindent 

\noindent We can get to (and change) our attributes simply by accessing these hash values directly (that is, by saying something like \$person-$>$\{address\}. Remember that we use this syntax because we're

\noindent dealing with a reference), but this is generally regarded as a bad idea. For starters, it requires us to know

\noindent the internal structure of the object and where and how the attributes are stored which, as an end-user, we should have no need or desire to fiddle with. Secondly, it doesn't give the object a chance to

\noindent examine the data you're giving it to make sure it makes sense. Instead, access to an attribute usually goes through a \textbf{method}.

\noindent 

\noindent \textit{Methods}

\noindent A method is anything you can tell the object to do. It could be something complicated, such as printing out address labels and reports, or something simple such as accessing an attribute. Those methods

\noindent directly related to attributes are called \textbf{get-set }methods, as they'll typically either \textbf{get }the current value of

\noindent the attribute, or \textbf{set }a new one.

\noindent 

\noindent The fact that methods are instructions for doing things may give you a clue as to how we represent them

\noindent in Perl -- methods in Perl are just subroutines. However, there's a special syntax called the 'arrow' operator (-$>$), which we use to call methods. So instead of getting the address attribute directly, as above, we're more likely to say something like this:

\noindent 

\noindent print "Address: ", \$person-$>$address(), "\textbackslash n";

\noindent 

\noindent We're also able to set an attribute (change its value) like this:

\noindent 

\noindent \$person-$>$address("Campus Mirabilis, Pisa, Italy");

\noindent 

\noindent Alternatively, we can call a method to produce an envelope for this object:

\noindent 

\noindent \$person-$>$print\_envelope();

\noindent 

\noindent This syntax \$object-$>$method(@arguments) 'invokes' the method, which just means that it calls the given subroutine -- in our examples this is either address or print\_envelope. We'll see how it's

\noindent done shortly.

\noindent 

\noindent 

\noindent \textit{Classes}

\noindent Our contact object and FTP session object are very different things -- they have different methods and attributes. While \$person-$>$date\_of\_birth() may make sense, you wouldn't expect, for instance

\noindent \$ftp\_session-$>$date\_of\_birth() to do anything sensible.

\noindent 

\noindent A \textbf{class }is the formal term for a \textit{type }of object. They define the methods an object can have, and how

\noindent those methods work. All objects in the Person class will have the same set of methods and possess the same attributes, and these will be different from the FTP class. An object is sometimes referred to as an

\noindent \textbf{instance }of a class, this just means that it's a specific thing created from a general category.

\noindent 

\noindent In Perl's object-oriented philosophy, a class is an ordinary package -- now let's start piecing this together:

\noindent 

  A method is a subroutine in a package. For instance, the date\_of\_birth method in the

\noindent Person class is merely the subroutine date\_of\_birth in the Person package.

\noindent 

  Blessing a scalar just means telling it what package to take its methods from. At that point, it's more than just a complex data structure, or scalar reference. It has attributes -- the data we've stored in the hash reference or elsewhere. It has methods -- the subroutines in its package, so it can be considered a fully-fledged object.

\noindent 

\noindent Classes can also have methods, in order to do things relevant to the whole class rather than individual objects. Instead of acting on an object, as you would by saying \$object-$>$method(), you act on the class: \$Person-$>$method(). An important thing to note is that Perl doesn't necessarily know whether

\noindent a given subroutine is a class method, an object method, or just an ordinary subroutine, so the

\noindent programmer has to do the checking himself.

\noindent 

\noindent Similarly, classes can have attributes that refer to the whole class -- in Perl these are just package

\noindent variables. For instance, we might have a population attribute in our Person class, which tells us how many Person objects are currently in existence.

\noindent 

\noindent One final note -- you'll probably have noticed that we capitalized \$Person. It's quite usual in Perl to capitalize all class names, so as to distinguish them from object names.

\noindent 

\noindent \textit{Polymorphism}

\noindent The word \textbf{polymorphism }comes from the Greek  , meaning 'many forms'. What it means in object-oriented programming is that a single method can do different things depending on the class of

\noindent the object that calls it. For instance, \$person-$>$address() would return the person's address, but

\noindent \$ftp\_session-$>$address() might return the IP address of the remote server. On the other hand,

\noindent \$object-$>$address() would \textit{have }to do the right thing according to which class \$object was in.

\noindent 

\noindent In some other OO programming languages, you have to ensure that the single method address can

\noindent cope with the various classes that make sense for it. Perl's OO model, on the other hand, neatly sidesteps the problem. This is because \$person-$>$address() and \$ftp\_session-$>$address() aren't a

\noindent single method at all. In fact, they're two different methods, because they refer to different subroutines.

\noindent 

\noindent One of these is the subroutine Person::address, and the other is the subroutine FTP::address. They're defined completely separately, in different packages, possibly even in different files. Since perl already knows what class each object belongs to, neither you nor perl need to do anything special to make the distinction. Perl looks at the object, finds the class it is in, and calls the subroutine in the appropriate package. This brings us on to\dots 

\noindent 

\noindent 

\noindent \textit{Encapsulation}

\noindent One of the nice things about object-oriented programming is that it hides complexity from. The user this is known as \textbf{encapsulation }(or \textbf{abstraction}). This means that you needn't care how the class is

\noindent structured, or how the attributes are represented in the object. You don't have to care how the methods work, or where they come from. You just use them.

\noindent 

\noindent This also means that the author of the class has complete freedom to change its internal workings at any time. As long as the methods have the same names and take the same arguments, all programs using the class should continue to work and produce the same results. That's as long as they use the method

\noindent interface as they should, rather than trying to poke at the data directly.

\noindent 

\noindent In this sense, working with objects is a little like driving a car. Our object, the car, has a set of attributes, such as the model, current speed, and amount of fuel in the tank. We can't get at these directly, but

\noindent some read-only methods like the speedometer and the fuel gauge expose them to us. It also provides us with some more methods and a well-defined interface to get it to do things.

\noindent 

\noindent We have a pedal to make it accelerate and one to make it brake, a stick to change gear, a hole to put

\noindent fuel into the tank, and so on. We don't actually need to know how the engine works if we're prepared to stick to using these methods, of course we do need to know what each of them does. We don't even

\noindent need to know the whereabouts of the fuel tank, we just put fuel in the appropriate hole. If we really want

\noindent to, we can take the hood off, look inside it, and fiddle with it. But then we only have ourselves to blame

\noindent if it breaks!

\noindent 

\noindent \textit{Inheritance}

\noindent Another property that makes OOP easy to use is its support for \textbf{inheritance}. Classes can be built quickly

\noindent by specifying how they differ from other classes. For example, humans inherit attributes from their parents, such as hair color and height, while Perl's classes inherit methods. If I inherit from a class, I receive the ability to call every method that class defines. If your class wants to implement a method differently, you define the method in my class. If you don't, you automatically get the method from the

\noindent parent class. The parent classes, which provide your class with the methods, are called \textbf{superclasses}, and

\noindent your class is a \textbf{subclass }of them.

\noindent 

\noindent The relationship between the classes can be described as an '\textbf{IS-A}' relationship. If you have a superclass

\noindent Animal, you may create a subclass Vertebrate. You could then say that a Vertebrate IS-A

\noindent Animal. In fact, the classification system for animals can be thought of as a series of IS-A relationships, with more specific subclasses inheriting properties of their superclasses:

\noindent 

\noindent 

\noindent \textit{Is Alive}

\noindent 

\noindent Animal Class

\noindent 

\noindent Superclass

\noindent 

\noindent 

\noindent 

\noindent IS A IS A

\noindent 

\noindent \textit{Has a backbone}

\noindent 

\noindent Animal: Vertebrate

\noindent Class

\noindent \textit{Has no backbone}

\noindent 

\noindent Animal: Invertebrate

\noindent Class

\noindent 

\noindent 

\noindent Subclasses

\noindent 

\noindent 

\noindent Here we see that vertebrates and invertebrates are both subclasses of a general animal class. They both

\noindent inherit the fact that they are alive, and so we need not specify this in the subclass. Next we could create

\noindent an Animal::Vertebrate::Mammal class, which would be a subclass of Animal::Vertebrate. We wouldn't need to specify that the mammal had a backbone or was alive, because these characteristics would be inherited from the superclass.

\noindent 

\noindent \textit{Constructors}

\noindent Objects have to come from somewhere, and in keeping with the principles of encapsulation, the user of

\noindent a class shouldn't be expected to put together an object himself. This would require knowledge of how the object is represented and what initialization is required. To take this responsibility away from the user, there's a class method that all classes should possess -- it's called the \textbf{constructor}.

\noindent 

\noindent As the name implies, this constructs and returns a new object. For this reason, it's usually called new(). We may pass arguments to the constructor, which it can then use to do the initial setup of the object. Sometimes these arguments are in the form of a hash, allowing us to create an object like this:

\noindent 

\noindent my \$galileo = Person-$>$new( surname =$>$ "Galilei", forename =$>$ "Galileo",

\noindent address =$>$ "9.81 Pisa Apts.", occupation =$>$ "bombadier",

\noindent );

\noindent 

\noindent There's also another syntax for calling methods, which you'll particularly see used with the constructor:

\noindent 

\noindent my \$galileo = new Person (...);

\noindent 

\noindent This is supported for the benefit of C++ programmers, as that language uses the method Class ()

\noindent syntax instead of the more Perl-like Class-$>$method(). Recognize it, but try and avoid using it.

\noindent 

\noindent The constructor will now check that the arguments are acceptable, do any conversion it requires, make up a hash reference, bless it and return it to us.

\noindent 

\noindent \textit{Destructors}

\noindent When the object is no longer in use -- when it's a lexical variable that goes out of scope, perl

\noindent automatically destroys it. However, before doing so, perl will attempt to call a method called DESTROY.

\noindent If the class provides this method, it should be responsible for any tasks that need to be performed before the object is disposed of. For instance, our FTP session object will want to ensure that it has closed the connection to the remote server.

\noindent 

\noindent Try It Out : Using a Net::FTP Object

\noindent We'll now use the Net::FTP module once again, to create an object that will let us get files from an

\noindent FTP site. In our case, we'll connect to CPAN and download the readme file:

\noindent 

\noindent \#!/usr/bin/perl

\noindent \# ftp.plx

\noindent use warnings; use strict; use Net::FTP;

\noindent 

\noindent 

\noindent my \$ftp = Net::FTP-$>$new("ftp.cpan.org")

\noindent or die "Couldn't connect: \$@\textbackslash n";

\noindent \$ftp-$>$login("anonymous");

\noindent \$ftp-$>$cwd("/pub/CPAN");

\noindent \$ftp-$>$get("README.html");

\noindent \$ftp-$>$close;

\noindent 

\noindent Network and firewalls permitting, this should retrieve the file -- although it may take some time:

\noindent 

\noindent $>$ \textbf{perl ftp.plx}

\noindent $>$ \textbf{dir README.html}

\noindent README\~{}1 HTM 2,902 ... ... README.html

\noindent $>$

\noindent 

\noindent \textit{How It Works}

\noindent After loading the Net::FTP module, we create ourselves an object:

\noindent 

\noindent my \$ftp = Net::FTP-$>$new("ftp.cpan.org")

\noindent or die "Couldn't connect: \$@\textbackslash n";

\noindent 

\noindent Our class is called Net::FTP, the same as the module -- this is because, as we mentioned above, a class

\noindent is just an ordinary package.

\noindent 

\noindent We create the object by calling the constructor, which is the class method new. This takes a number of arguments: a remote machine to connect to and a hash specifying things like whether we have a

\noindent firewall, which port to connect to, whether we want debugging information, and so on. These arguments will become the attributes of the object. If we don't specify them, the constructor comes up with some sensible defaults for us. In our case, the defaults are fine, so we just need to supply a remote machine -- we'll use the CPAN server, ftp.cpan.org.

\noindent 

\noindent Now  we know  that a Perl  method  is  just  a  subroutine.  A  method  call,  with  the  arrow  operator,  is

\noindent like  calling  a subroutine,  but with  two  important  differences.  First  inheritance  means  that  perl  won't immediately  complain  if the subroutine  doesn't  exist,  but  instead  will  look  first  to  see if  the  subroutine

\noindent is  in each of the superclasses.  Second,  what  goes  before  the  arrow  becomes  the first  parameter  of the subroutine.

\noindent 

\noindent So, we could, if we wanted to, rewrite our call to the constructor like this:

\noindent 

\noindent my \$ftp = Net::FTP::new("Net::FTP","ftp.cpan.org")

\noindent or die "Couldn't connect: \$@\textbackslash n";

\noindent 

\noindent However, although this shows us exactly what's happening in excruciating detail, it's rather unwieldy, and disguises the fact that we're dealing with objects.

\noindent 

\noindent When we call the constructor, it takes our argument (the remote host) and stashes it away in a hash -- encapsulation means we don't need to know how or where. Then it takes a reference to that hash, blesses the reference, and returns it to us. That blessed reference is our new object (our FTP session), and we're now ready to do things with it:

\noindent 

\noindent \$ftp-$>$login("anonymous");

\noindent 

\noindent 

\noindent First of all, we have to log in to the server. The usual way of getting things from an FTP server is by

\noindent logging in with a username of 'anonymous' and your email address as the password. The login method tells the object to issue the appropriate login commands. As before, this is an ordinary subroutine. It

\noindent could be written like this:

\noindent 

\noindent Net::FTP::login(\$ftp,"anonymous");

\noindent 

\noindent How did perl know that it should use Net::FTP::login rather than any other login? Well, when

\noindent our constructor blessed the reference, it gave the reference knowledge of where to find the methods. To quote from the perlobj documentation, "an object is just a reference that happens to know which

\noindent class it belongs to".

\noindent 

\noindent Since perl takes care of passing the object to the subroutine as the first parameter, the sub automatically receives all the data it needs. This means we can easily have multiple objects doing different things:

\noindent 

\noindent my \$ftp1 = Net::FTP-$>$new("ftp.cpan.org");

\noindent my \$ftp2 = Net::FTP-$>$new("ftp.wrox.com");

\noindent \$ftp1-$>$login("anonymous");

\noindent 

\noindent The object \$ftp1 is just a blessed reference to a hash, and that hash contains all the data about the connection to CPAN, like the settings, the filehandles, and anything else that Net::FTP needs to store. These are the object's attributes. Everything we know about the connection is bundled into that object. The important thing to note is that it's completely independent from \$ftp2, which is another hash containing another set of data about a different connection. Hence, the method call \$ftp1-$>$login() has no impact on the other connection at all.

\noindent 

\noindent What's the difference between a class method and an object method? Well, nothing really as far as perl

\noindent is concerned. Perl doesn't make a distinction. In theory, there's nothing to stop you saying something

\noindent like Net::FTP-$>$login() or \$ftp-$>$new(). However, because a class method expects a string as it's first parameter and an object method expects an object, they're likely to get confused if you give them

\noindent the wrong thing. Alternatively, if the class author is particularly conscientious, the method will check whether it is being called on a class or an object and take the appropriate action. On the other hand, don't expect it to work:

\noindent 

\noindent $>$ \textbf{perl -e 'use Net::FTP; Net::FTP-$>$login'}

\noindent Can't use string ("Net::FTP") as a symbol ref while "strict refs" in use at

\noindent /usr/local/lib/perl5/site\_perl/5.6.0/Net/FTP.pm line 238.

\noindent $>$

\noindent 

\noindent Here, the login() method was called as a class method. As we saw with new above, when you call a

\noindent class method, perl calls the subroutine with the class name as the first parameter, in effect it's doing this:

\noindent 

\noindent Net::FTP::login("Net::FTP");

\noindent 

\noindent Because login was written to be an object method rather than a class method, it's expecting to get an object rather than the name of a class. Now an object, as we know, is just a blessed reference. When login tries to use the parameter it's got as a reference, it finds out that it's actually a string, and perl gives the error message you see above.

\noindent 

\noindent So, while perl calls object and class methods in exactly the same way, because a method is usually written to be one rather than another, it's likely to blow up if you call it inappropriately.

\noindent 

\noindent 

\noindent Anyway, back to our example! After logging in, we change directory and get the file:

\noindent 

\noindent 

\noindent \$ftp-$>$cwd("/pub/CPAN");

\noindent \$ftp-$>$get("README.html");

\noindent 

\noindent cwd and get are two more methods our object supplies. The object has a huge number of methods, due

\noindent to the fact that it has a long chain of inheritance. It inherits from a superclass, which inherits from a superclass, which inherits from a superclass and so on. However, there are some methods which Net::FTP defines directly and which you should know about. They mainly relate directly to FTP commands -- here is an incomplete list of them:

\noindent 

\noindent 

\noindent \textbf{Method Name Behaviour}

\noindent 

\noindent \$session-$>$login(\$login,\$passwd) Logs into the server with the given username and password.

\noindent 

\noindent \$session-$>$type(\$type)

\noindent \$session-$>$ascii()

\noindent \$session-$>$binary()

\noindent 

\noindent Set the transfer type to ASCII or binary: this is quite similar to Perl's binmode operator.

\noindent 

\noindent \$session-$>$rename(\$old,\$new) Rename a file.

\noindent 

\noindent \$session-$>$delete(\$file) Delete a file.

\noindent 

\noindent \$session-$>$cwd(\$directory) Change directory.

\noindent 

\noindent \$session-$>$pwd() Give the name of the current directory.

\noindent 

\noindent \$session-$>$ls() List the current directory.

\noindent 

\noindent \$session-$>$get(\$remote, \$local,

\noindent \$offset)

\noindent 

\noindent Get a file from the remote server.

\noindent 

\noindent \$session-$>$put(\$local, \$remote) Put a file to the remote server.

\noindent 

\noindent There are also some get-set methods that will affect the object's attributes. For instance, the \$session-

\noindent $>$hash() method controls an attribute that determines whether or not to print \# signs after every 1024

\noindent bytes transferred.

\noindent 

\noindent After we've called the get method to get our file, we'll call the close method to shut down the connection to the server.

\noindent 

\noindent \$ftp-$>$close;

\noindent 

\noindent Again, this is equivalent to Net::FTP::close(\$ftp), but more convenient.

\noindent 

\noindent So, we've used our first class. Hopefully, you can see that using objects and classes in Perl is just as easy

\noindent as using subroutines. In fact, it's easier -- perl not only takes care of finding out where to find the subroutine you're trying to call, but it also takes care of passing a whole bunch of data to the subroutine for you.

\noindent 

\noindent Because this all goes on behind our backs, we can happily pretend that an object contains a bunch of methods that act on it, and it alone. In fact, it doesn't -- it only contains information regarding where to find methods that can act on any object in that class.

\noindent 

\noindent 

\noindent Rolling Your Own

\noindent 

\noindent We've now seen how to use a class and an object. Let's now see how to make our own classes. As an example, we'll implement the Person class we used in our definitions.

\noindent 

\noindent As we mentioned above, a class is just a package -- nothing more, nothing less. So the simplest class looks like this:

\noindent 

\noindent package Person;

\noindent 

\noindent That's it. Of course, it has nothing -- no methods, no attributes, no constructor, nothing. It's a totally

\noindent empty class. Usually, you'll want to put your class into a module. It's not necessary by any means, but it gets the implementation out of the way. So, let's create a module, and put the following in the file Person1.pm

\noindent 

\noindent package Person;

\noindent \# Class for storing data about a person

\noindent \#person1.pm

\noindent use warnings;

\noindent use strict;

\noindent 

\[1;\] 


\noindent Remember that we need the  TRUE value  as  the  last  thing  to  tell  Perl  that  everything  went  OK

\noindent when loading the file.  Now  in  a  separate  program,  we can  say  use Person and  start  using  the class.  However,  we can't  create  any  objects  yet,  because  we don't  have  a  constructor.  So,  how  do

\noindent we write a constructor?

\noindent 

\noindent Well,  what does our constructor create? It creates an object, which is a blessed reference. Before we go any further, then, let's have a look at what bless is and what it does.

\noindent 

\noindent Bless You, My Reference

\noindent The bless operator takes a reference and turns it into an object. The way it does that is simple: It changes the type of the reference. Instead of being an array reference or a hash reference, perl now thinks of it as a Person reference (or whatever other class we bless the reference into).

\noindent 

\noindent As we saw in Chapter 7, we can use the ref operator to tell us what type of reference we have. To refresh your memory:

\noindent 

\noindent \#!/usr/bin/perl

\noindent \# reftypes.plx

\noindent use warnings;

\noindent use strict;

\noindent 

\noindent my \$a = [];

\noindent my \$b = \{\};

\noindent my \$c = \textbackslash 1;

\noindent my \$d = \textbackslash \$c;

\noindent print '\$a is a ', ref \$a, " reference\textbackslash n";

\noindent print '\$b is a ', ref \$b, " reference\textbackslash n";

\noindent print '\$c is a ', ref \$c, " reference\textbackslash n";

\noindent print '\$d is a ', ref \$d, " reference\textbackslash n";

\noindent 

\noindent 

\noindent $>$\textbf{perl reftypes.plx}

\noindent \$a is a ARRAY reference

\noindent \$b is a HASH reference

\noindent \$c is a SCALAR reference

\noindent \$d is a REF reference

\noindent $>$

\noindent 

\noindent Now let's see what happens when we use bless. The syntax of bless is:

\noindent 

\noindent 

\noindent bless( $<$reference$>$, $<$package$>$ );

\noindent 

\noindent If the package isn't given, the reference is blessed into the current package. Let's bless a reference into the Person package:

\noindent 

\noindent \#!/usr/bin/perl

\noindent \# bless1.plx use warnings; use strict;

\noindent 

\noindent my \$a = \{\};

\noindent 

\noindent print '\$a is a ', ref \$a, " reference\textbackslash n";

\noindent 

\noindent bless(\$a, "Person");

\noindent 

\noindent print '\$a is a ', ref \$a, " reference\textbackslash n";

\noindent 

\noindent $>$\textbf{perl bless1.plx}

\noindent \$a is a HASH reference

\noindent \$a is a Person reference

\noindent $>$

\noindent 

\noindent Okay, so we've changed \$a into a Person reference. So what just happened?

\noindent 

\noindent Actually, nothing changed in the structure of \$a at all. It's still a hash reference, and we can still dereference it -- or add, access, and delete entries in the hash, and so on. It still has the same keys and values. Nothing magical has happened.

\noindent 

\noindent But \$a is now a reference with knowledge of which package it belongs to. If we try and call a method

\noindent on it, perl now knows that it should look in the Person package for a definition of that method. It's become an object.

\noindent 

\noindent What if we bless it again? What happens then? Let's try it.

\noindent 

\noindent 

\noindent \#!/usr/bin/perl

\noindent \# bless2.plx

\noindent use warnings;

\noindent use strict;

\noindent 

\noindent my \$a = \{\};

\noindent 

\noindent 

\noindent print '\$a is a ', ref \$a, " reference\textbackslash n";

\noindent 

\noindent bless(\$a, "Person");

\noindent print '\$a is a ', ref \$a, " reference\textbackslash n";

\noindent 

\noindent bless(\$a, "Animal::Vertebrate::Mammal");

\noindent print '\$a is a ', ref \$a, " reference\textbackslash n";

\noindent 

\noindent $>$ \textbf{perl bless2.plx}

\noindent \$a is a HASH reference

\noindent \$a is a Person reference

\noindent \$a is a Animal::Vertebrate::Mammal reference

\noindent $>$

\noindent 

\noindent All that's happened is we've once again changed what type of reference it is. We've changed where perl should look if any methods are called on the reference. Note that at this stage we haven't even defined

\noindent an Animal::Vertebrate::Mammal package, but that's OK because we're not going to call any

\noindent methods yet -- if we did, they would surely fail.

\noindent 

\noindent Again, the internal structure of that reference hasn't changed. It's still a hash reference with the same keys and values. You usually don't want to bless an object that's already been blessed. This is because something that was originally a Person may have different attributes to what the new class

\noindent expects it to have when methods are called. Worse still, the program using the object could well try and

\noindent call a method that was fine in the old class but doesn't exist in the new one -- attempting to magically turn a person into an FTP session can only have undesirable (and pretty weird) results.

\noindent 

\noindent Storing Attributes

\noindent 

\noindent We've got an object. Before we look at methods, let's examine attributes. An attribute is, as we defined it

\noindent at the start of this chapter, something we know about the object. In other words, it's a piece of data that belongs to this particular object. How do we store this data, then?

\noindent 

\noindent Well, this is what the reference is for, if we store our data in the reference, our object carries around

\noindent both a set of data unique to it, plus knowledge of where to find methods to act on that data. If we know that our object is only going to contain one attribute (one piece of data), we could conceivably use a scalar reference, like this:

\noindent 

\noindent my \$attribute = "green"; my \$object = \textbackslash \$attribute; bless(\$object, "Simple");

\noindent 

\noindent Now we have a nice simple object that stores a single attribute contained in the Simple class. We can access and change the attribute just as we'd work with an ordinary scalar reference:

\noindent 

\noindent \$attribute = \$\$object;

\noindent \$\$object = "red";

\noindent 

\noindent This is nice and simple, but it's not very flexible. Similarly, we could have an array reference and

\noindent bless that to turn it into an object, which is slightly more flexible. We can access attributes as elements

\noindent in the array, and we can add and delete attributes by using array operations. If we are storing a set of unnamed data, this is perfectly adequate.

\noindent 

\noindent 

\noindent However, for maximum flexibility, we can use a hash to give names to our attributes:

\noindent 

\noindent my \$object = \{

\noindent surname =$>$ "Galilei", forename =$>$ "Galileo",

\noindent address =$>$ "9.81 Pisa Apts.",

\noindent occupation =$>$ "bombadier",

\noindent \};

\noindent bless \$object, "Person";

\noindent 

\noindent This allows us easy access to the individual attributes, as if we were carrying a bunch of variables around with us. Therefore, we generally use a hash reference for any non-trivial class.

\noindent 

\noindent The Constructor

\noindent 

\noindent We're now ready to create objects. Let's put this knowledge into a constructor, and put a constructor into our currently empty Person class.

\noindent 

\noindent Try It Out : Our First Constructor

\noindent To construct an object, we make a hash reference, and bless it into the class. That's all we need to do:

\noindent 

\noindent package Person;

\noindent \# Class for storing data about a person

\noindent \#person2.pm

\noindent use warnings;

\noindent use strict;

\noindent 

\noindent sub new \{

\noindent my \$self = \{\};

\noindent bless (\$self, "Person");

\noindent return \$self;

\noindent \}

\noindent 

\[1;\] 


\noindent Now we can use our Person class to create an object:

\noindent 

\noindent \#/usr/bin/perl

\noindent \# persontest.plx

\noindent use warnings;

\noindent use strict;

\noindent use Person2;

\noindent 

\noindent my \$person = Person-$>$new();

\noindent 

\noindent which should execute without any errors!

\noindent 

\noindent \textit{How It Works}

\noindent Our constructor does a simple job, and does it well. First, we create our hash reference:

\noindent 

\noindent my \$self = \{\};

\noindent 

\noindent 

\noindent \$self is the traditional name for an object when it's being manipulated by methods inside the class.

\noindent Now we'll turn it into an object by telling it which class it belongs to:

\noindent 

\noindent bless (\$self, "Person");

\noindent 

\noindent Finally, we send the object back:

\noindent 

\noindent return \$self;

\noindent 

\noindent Excellent. Now let's see how we can improve this.

\noindent 

\noindent \textit{Considering Inheritance}

\noindent The first thing we've  got to  think  about  is  inheritance.  It's  possible  that  someone,  someday  will

\noindent want to  inherit from this  class,  and  we  won't  necessarily  get  told  about  it.  If  they  don't  provide  their

\noindent own  constructor,  they'll  get ours,  and  as  things  stand,  that'll  produce  an  object  blessed  into  our  class --

\noindent not theirs.

\noindent 

\noindent We really need to remove the hard-wired "Person" in our constructor and replace it with the called class. How do we know what the called class is though? Well, the thing to remember is that Perl translates Class-$>$new() into new("Class"). We know what class the user wants, because it's the first argument to the constructor. All we need to do is take that argument and use that as the class to bless into. So here's a more general constructor that takes inheritance into account:

\noindent 

\noindent sub new \{

\noindent my \$class = shift;

\noindent my \$self = \{\}; bless(\$self, \$class); return \$self;

\noindent \}

\noindent 

\noindent As usual, shift without any arguments means shift @\_. It takes the first element of the argument

\noindent array. This gives us the first thing we were passed, the class name. We can therefore use this to bless our reference, without needing to hard-code the name in.

\noindent 

\noindent \textit{Providing Attributes}

\noindent Now let's make one more enhancement. At the moment, we can create a completely anonymous

\noindent Person with no attributes at all. We can give the end-user of the class the opportunity to specify some attributes when the Person is created.

\noindent 

\noindent Try It Out : Initializing Attributes In The Constructor

\noindent As before, we're going to store the data in the hash reference. We'll take the data as arguments to the constructor. Ideally, we'll want the constructor to be called something along these lines:

\noindent 

\noindent my \$object = Person-$>$new ( surname =$>$ "Galiiei", forename =$>$ "Galileo",

\noindent address =$>$ "9.81 Pisa Apts.",

\noindent occupation =$>$ "bombardier"

\noindent );

\noindent 

\noindent 

\noindent This is the easiest syntax for the user, because it allows them to specify the attributes in any order and

\noindent give as many or as few as they want. It's also a lot easier to use and remember than if we make them use

\noindent a list like this:

\noindent 

\noindent my \$object = Person-$>$new ("Galilei","Galileo","9.81 Pisa Apts.","bombardier");

\noindent 

\noindent In  fact,  it's the easiest  syntax  for  us,  too.  Since  we  want  our  attributes  stored  in  a  hash,  and  the  key- value  syntax we propose above  is  a  hash,  all  we have  to  do  is  place  the arguments  straight  into  our hash reference:

\noindent 

\noindent my \$self = \{@\_\};

\noindent 

\noindent Let's plug this into our package:

\noindent 

\noindent package Person;

\noindent \# Class for storing data about a person

\noindent \#person3.pm use warnings; use strict;

\noindent 

\noindent sub new \{

\noindent my \$class = shift;

\noindent my \$self = \{@\_\};

\noindent bless(\$self, \$class);

\noindent return \$self;

\noindent \}

\noindent 

\[1;\] 


\noindent \textit{How It Works}

\noindent What have we done? Well, now when we call the constructor, Perl sees something like this:

\noindent 

\noindent my \$object = Person::new("Person", "surname", "Galilei", "forename",  "Galileo", "address", "9.81 Pisa Apts.", "occupation","bombardier"

\noindent );

\noindent 

\noindent The first line of the constructor takes up the class name as before:

\noindent 

\noindent my \$class = shift;

\noindent 

\noindent Now what's left in the argument array @\_ is what we specified when we called the constructor:

\noindent 

\noindent @\_=(

\noindent "surname", "Galilei", "forename",  "Galileo", "address", "9.81 Pisa Apts.", "occupation","bombardier"

\noindent );

\noindent 

\noindent 

\noindent This is what we put verbatim into our hash reference:

\noindent 

\noindent my \$self = \{@\_\};

\noindent 

\noindent Our hash now contains all the attributes we provided. As usual, it's blessed and returned to the caller.

\noindent 

\noindent 

\noindent \textit{my \$self creates a lexical variable, which is destroyed once the subroutine ends. Doesn't this mean that all our attributes will get wiped out too? Nope -- this is exactly why we use a reference.}

\noindent \textit{Perl will never destroy data while a reference to it exists (see the section on 'Reference Counting' in}

\noindent \textit{Chapter 7) our data will persist everywhere the object goes.}

\noindent 

\noindent 

\noindent We've now got a fully featured constructor. We've taken some initial data and constructed an object out

\noindent of it, storing the data as attributes in the object. Now it's time to add some methods so we can actually do something with it!

\noindent 

\noindent Creating Methods

\noindent 

\noindent Our constructor was a class method; creating an object method will be very similar. In the same way that a class method gets passed the name of the class as the first argument, an object method is just a subroutine that gets passed the object as the first argument.

\noindent 

\noindent Try It Out : A Simple Accessor

\noindent Let's create a method to return the surname of the person. This directly accesses an attribute --

\noindent sometimes called an \textbf{accessor method}. Remember that the surname attribute is just an entry in the hash, referenced by the object -- so what does this involve? We'll need to:

\noindent 

\noindent ? receive the object being passed to us

\noindent 

\noindent ? extract the 'surname' entry from the hash

\noindent 

\noindent ? pass it back to the caller

\noindent 

\noindent Here's how we'd code it:

\noindent 

\noindent 

\noindent sub surname \{

\noindent my \$self = shift;

\noindent my \%hash = \%\$self;

\noindent return \$hash\{surname\}

\noindent \}

\noindent 

\noindent However, with the techniques we learned in Chapter 7 for directly accessing values in a hash reference, we can trim it down a bit and add it into our class:

\noindent 

\noindent package Person;

\noindent \# Class for storing data about a person

\noindent \#person4.pm use warnings; use strict;

\noindent 

\noindent 

\noindent sub new \{

\noindent my \$class = shift;

\noindent my \$self = \{@\_\};

\noindent bless(\$self, \$class);

\noindent return \$self;

\noindent \}

\noindent 

\noindent sub surname \{

\noindent my \$self = shift;

\noindent return \$self-$>$\{surname\}

\noindent \}

\noindent 

\[1;\] 


\noindent Now we can create an object with some attributes and retrieve the attributes again:

\noindent 

\noindent 

\noindent \#!/usr/bin/perl

\noindent \# accessor1.plx

\noindent use Person4;

\noindent 

\noindent my \$object = Person-$>$new (

\noindent surname =$>$ "Galilei",

\noindent forename =$>$ "Galileo",

\noindent address =$>$ "9.81 Pisa Apts.",

\noindent occupation =$>$ "bombadier"

\noindent );

\noindent print "This person's surname: ", \$object-$>$surname, "\textbackslash n";

\noindent 

\noindent If all's well, we should be told the surname:

\noindent 

\noindent \textbf{$>$ perl accessor1.plx}

\noindent This person's surname: Galilei

\noindent $>$

\noindent 

\noindent \textit{How It Works}

\noindent Our method is a very simple one -- it takes an object and extracts an attribute from that object's data store. First, we use shift to get the object passed to us:

\noindent 

\noindent my \$self = shift;

\noindent 

\noindent Then we take out the relevant hash entry and pass it back:

\noindent 

\noindent return \$self-$>$\{surname\}

\noindent 

\noindent Don't confuse the arrow used here for accessing parts of a reference with the arrow used as a method call. When accessing a reference, there will always be some kind of brackets at the end of the arrow:

\noindent 

\noindent \$reference-$>$\{surname\}; \# Accesses a hash reference

\noindent \$reference-$>$[3]; \# Accesses an array reference

\noindent \$reference-$>$(); \# Accesses a function reference

\noindent 

\noindent 

\noindent When calling a method, there will be a name following the arrow:

\noindent 

\noindent \$reference-$>$surname;

\noindent 

\noindent There may be brackets after that if parameters are being passed to the method or if the programmer wants to use () to be unambiguous:

\noindent 

\noindent \$reference-$>$surname();

\noindent 

\noindent 

\noindent \textbf{A direct access to a reference will always have some kind of bracket after the arrow,}

\noindent \textbf{no matter what type of reference. A method call, whether on a class or on an object,}

\noindent \textbf{will have the name of the method directly after the arrow with no brackets in between.}

\noindent 

\noindent 

\noindent So while our method is called with \$object-$>$surname, the surname entry in the hash is accessed

\noindent with \$self-$>$\{surname\}. Thankfully, the only time you'll see this is when you're creating accessors.

\noindent An accessor is the only place where it's safe to directly access part of an object's reference. Accessors are the interface through which everything talks to the attributes.

\noindent 

\noindent The reasoning behind this is, as before, inheritance. If someone inherits a method from your class but changes the internal representation of the object slightly, they'll be stuck if some of your methods access the object directly. For instance, if an inheriting class decides that it should take data from a file on disk

\noindent but your methods persist in accessing data inside a hash reference, your methods won't be usable. However, if you always use accessor methods, the inheriting class can provide its own accessors, which will use the file instead of a hash, and all will be well.

\noindent 

\noindent \textit{Distinguishing Class and Object Methods}

\noindent Now, we've mentioned a few times that Perl doesn't distinguish between class and object methods. What

\noindent if we want to? Our surname method only makes sense with an object, and not with a class, so if our surnames are called incorrectly, we want to shout about it.

\noindent 

\noindent Well, the thing that lets us know how we're being called is that first parameter. If we have a string, we're being called as a class method. If we have a reference, we're being called as an object method. So, let's make one more alteration to our accessor method to trap incorrect usage.

\noindent 

\noindent Try It Out : Checking Usage

\noindent If our accessor is called with something that isn't an object, we'll give an error message and blame the programmer:

\noindent 

\noindent package Person;

\noindent \# Class for storing data about a person

\noindent \#person5.pm

\noindent use warnings;

\noindent use strict;

\noindent use Carp;

\noindent 

\noindent sub new \{

\noindent my \$class = shift;

\noindent my \$self = \{@\_\};

\noindent bless(\$self, \$class);

\noindent return \$self;

\noindent \}

\noindent 

\noindent 

\noindent sub surname \{

\noindent my \$self = shift;

\noindent unless (ref \$self) \{

\noindent croak "Should call surname() with an object, not a class";

\noindent \}

\noindent return \$self-$>$\{surname\}

\noindent \}

\noindent 

\[1;\] 


\noindent Now if we add the following line to the end of our accessor1.plx file:

\noindent 

\noindent 

\noindent Person-$>$surname;

\noindent 

\noindent we should generate the following complaint:

\noindent 

\noindent $>$\textbf{perl accessor1.plx}

\noindent This object's surname: Galilei

\noindent Should call surname() with an object, not a class at accessor.plx line 12

\noindent $>$

\noindent 

\noindent \textit{How It Works}

\noindent We use the ref operator to make sure that what we're being passed is actually a reference. We could be more stringent, and ensure that it's an object in the Person class, but again that would break inheritance.

\noindent 

\noindent \textit{Get-Set Methods}

\noindent As well as getting the value of an attribute, we may well want to set or change it. The syntax we'll use is

\noindent as follows:

\noindent 

\noindent 

\noindent print "Old address: ", \$person-$>$address(), "\textbackslash n";

\noindent \$person-$>$address("Campus Mirabilis, Pisa, Italy");

\noindent print "New address: ", \$person-$>$address(), "\textbackslash n";

\noindent 

\noindent This kind of accessor is called a get-set method, because we can use it to both get and set the attribute. Turning our current read-only accessors into accessors that can also set the value is simple. Let's create a get-set method for address:

\noindent 

\noindent sub address \{

\noindent my \$self = shift;

\noindent unless (ref \$self) \{

\noindent croak "Should call address() with an object, not a class";

\noindent \}

\noindent 

\noindent \# Receive more data

\noindent my \$data = shift;

\noindent \# Set the address if there's any data there.

\noindent \$self-$>$\{address\} = \$data if defined \$data;

\noindent 

\noindent return \$self-$>$\{address\}

\noindent \}

\noindent 

\noindent 

\noindent If we don't particularly want to trap calling the method as a class method (since it'll generate an error

\noindent when we try to access the hash entry, anyway), we can write really miniature get-set methods like this:

\noindent 

\noindent sub address  \{ \$\_[0]-$>$\{address \}=\$\_[1] if defined \$\_[1]; \$\_[0]-$>$\{address \} \} sub surname  \{ \$\_[0]-$>$\{surname \}=\$\_[1] if defined \$\_[1]; \$\_[0]-$>$\{surname \} \} sub forename \{ \$\_[0]-$>$\{forename\}=\$\_[1] if defined \$\_[1]; \$\_[0]-$>$\{forename\} \}

\noindent 

\noindent While that's fine for getting classes up and running quickly, writing out the get-set method in full as above allows us to easily extend it in various ways, like testing the validity of the data, doing any notification we need to when the data changes, and so on.

\noindent 

\noindent \textit{Class Attributes}

\noindent Classes can have attributes, too. Instead of being entries in a hash, they're variables in a package. Just like object attributes it's a really good idea to access them through get-set methods, but since they're

\noindent ordinary variables, our methods are a lot simpler. Let's use a class attribute to keep score of how many times we've created a Person object. We'll call our attribute \$Person::Population, and we'll get

\noindent the current value of it via the method headcount.

\noindent 

\noindent Try It Out : Adding A Class Attribute

\noindent A class attribute is a package variable, and an accessor method just returns or sets the value of that variable. Here, we make our accessor method read-only, to stop the end user changing it and confusing his own code:

\noindent 

\noindent package Person;

\noindent \# Class for storing data about a person

\noindent \#person6.pm

\noindent use warnings;

\noindent use strict;

\noindent use Carp;

\noindent 

\noindent my \$Population = 0;

\noindent 

\noindent sub new \{

\noindent my \$class = shift;

\noindent my \$self = \{@\_\};

\noindent bless(\$self, \$class);

\noindent \$Population++;

\noindent return \$self;

\noindent \}

\noindent 

\noindent \# Object accessor methods

\noindent sub address  \{ \$\_[0]-$>$\{address \}=\$\_[1] if defined \$\_[1]; \$\_[0]-$>$\{address \} \}

\noindent sub surname  \{ \$\_[0]-$>$\{surname \}=\$\_[1] if defined \$\_[1]; \$\_[0]-$>$\{surname \} \} sub forename \{ \$\_[0]-$>$\{forename\}=\$\_[1] if defined \$\_[1]; \$\_[0]-$>$\{forename\} \} sub phone\_no\{ \$\_[0]-$>$\{phone\_no\}=\$\_[1] if defined \$\_[1]; \$\_[0]-$>$\{phone\_no\} \} sub occupation  \{

\noindent \$\_[0]-$>$\{occupation\}=\$\_[1] if defined \$\_[1]; \$\_[0]-$>$\{occupation\}

\noindent \}

\noindent 

\noindent \# Class accessor methods

\noindent sub headcount \{ \$Population \}

\noindent 

\[1;\] 


\noindent 

\noindent Now as we create new objects, the population increases:

\noindent 

\noindent \#!/usr/bin/perl

\noindent \# classatr1.plx use warnings; use strict;

\noindent use Person6;

\noindent 

\noindent print "In the beginning: ", Person-$>$headcount, "\textbackslash n";

\noindent my \$object = Person-$>$new (

\noindent surname =$>$ "Galilei",

\noindent forename =$>$ "Galileo",

\noindent address =$>$ "9.81 Pisa Apts.",

\noindent occupation =$>$ "bombadier"

\noindent );

\noindent print "Population now: ", Person-$>$headcount, "\textbackslash n";

\noindent 

\noindent my \$object2 = Person-$>$new ( surname =$>$ "Einstein", forename =$>$ "Albert",

\noindent address =$>$ "9E16, Relativity Drive", occupation =$>$ "Plumber"

\noindent );

\noindent print "Population now: ", Person-$>$headcount, "\textbackslash n";

\noindent 

\noindent $>$\textbf{perl classatr1.plx}

\noindent In the beginning: 0

\noindent Population now: 1

\noindent Population now: 2

\noindent $>$

\noindent 

\noindent \textit{How It Works}

\noindent There's actually nothing OO specific about this example. All we're doing is taking advantage of the way

\noindent Perl's scoping works. A lexical variable can be seen and used by anything in the current scope and inside any brackets. So, naturally enough, with:

\noindent 

\noindent Package Person;

\noindent my \$Population;

\noindent 

\noindent sub headline \{ \$Population \}

\noindent 

\noindent the package variable \$Population is declared at the top of the package and is therefore visible everywhere in the package. Even though we call headcount from another package, it accesses a variable in its own package.

\noindent 

\noindent Similarly, when we increase it as part of new, we're accessing a variable in the same package. Since it's a package variable, it stays around for as long as the package does, which is why it doesn't lose its value when we do things in our main program.

\noindent 

\noindent Let's make one more addition. We'll allow our main program to go over all of the names of people in our contacts database, and we'll have a class method to give us an array of the objects created. Instead

\noindent of keeping a separate variable for the population, we'll re-implement \$Population in terms of the scalar value of that array:

\noindent 

\noindent 

\noindent package Person;

\noindent \# Class for storing data about a person

\noindent \#person7.pm

\noindent use warnings;

\noindent use strict;

\noindent use Carp;

\noindent 

\noindent my @Everyone;

\noindent 

\noindent sub new \{

\noindent my \$class = shift;

\noindent my \$self = \{@\_\};

\noindent bless(\$self, \$class);

\noindent push @Everyone, \$self;

\noindent return \$self;

\noindent \}

\noindent 

\noindent \# Object accessor methods

\noindent sub address  \{ \$\_[0]-$>$\{address \}=\$\_[1] if defined \$\_[1]; \$\_[0]-$>$\{address \} \}

\noindent sub surname  \{ \$\_[0]-$>$\{surname \}=\$\_[1] if defined \$\_[1]; \$\_[0]-$>$\{surname \} \} sub forename \{ \$\_[0]-$>$\{forename\}=\$\_[1] if defined \$\_[1]; \$\_[0]-$>$\{forename\} \} sub phone\_no\{ \$\_[0]-$>$\{phone\_no\}=\$\_[1] if defined \$\_[1]; \$\_[0]-$>$\{phone\_no\} \} sub occupation  \{

\noindent \$\_[0]-$>$\{occupation\}=\$\_[1] if defined \$\_[1]; \$\_[0]-$>$\{occupation\}

\noindent \}

\noindent 

\noindent \# Class accessor methods

\noindent sub headcount \{ scalar @Everyone \}

\noindent sub everyone  \{ @Everyone \}

\noindent 

\[1;\] 


\noindent Note that we're pushing one reference to the data onto the array, and we return another reference. There are now two references to the same data, rather than two copies of the data. This becomes important

\noindent when it comes to destruction. Anyway, this time we can construct our objects and look over them:

\noindent 

\noindent \#!/usr/bin/perl

\noindent \# classatr2.plx use warnings; use strict;

\noindent use Person7;

\noindent 

\noindent print "In the beginning: ", Person-$>$headcount, "\textbackslash n";

\noindent my \$object = Person-$>$new (

\noindent surname =$>$ "Galilei",

\noindent forename =$>$ "Galileo",

\noindent address =$>$ "9.81 Pisa Apts.",

\noindent occupation =$>$ "bombadier"

\noindent );

\noindent print "Population now: ", Person-$>$headcount, "\textbackslash n";

\noindent 

\noindent my \$object2 = Person-$>$new ( surname =$>$ "Einstein", forename =$>$ "Albert",

\noindent 

\noindent address =$>$ "9E16, Relativity Drive", occupation =$>$ "Plumber"

\noindent );

\noindent 

\noindent 

\noindent print "Population now: ", Person-$>$headcount, "\textbackslash n";

\noindent 

\noindent print "\textbackslash nPeople we know:\textbackslash n";

\noindent for my \$person(Person-$>$everyone) \{

\noindent print \$person-$>$forename, " ", \$person-$>$surname, "\textbackslash n";

\noindent \}

\noindent 

\noindent $>$\textbf{perl classatr2.plx}

\noindent In the beginning: 0

\noindent Population now: 1

\noindent Population now: 2

\noindent 

\noindent People we know: Galileo Galilei Albert Einstein

\noindent $>$

\noindent 

\noindent Normally, you won't want to do something like this. It's not the class's business to know what's being done with the objects it creates. Since we know that in these examples we'll be putting all the Person objects into a database, it's reasonable to get the whole database with a single method. However, this isn't a general solution -- people may not use the objects they create, or may use them in multiple databases, or in other ways you haven't thought of. Let the user keep copies of the object themselves.

\noindent 

\noindent \textit{Privatizing Your Methods}

\noindent The things we did with our class attributes in new in the two examples above were a bit naughty. We directly accessed the class variables, instead of going through an accessor method. If another class wants

\noindent to inherit, it has to make sure it too carries a package variable of the same name in the same way.

\noindent 

\noindent What we usually do in these situations is to put all the class-specific parts into a separate method and use that method internally in the class. Inheriting classes can then replace these \textbf{private methods }with

\noindent their own implementations. To mark a method as private, for use only inside the class, it's customary to begin the method's name with an underscore. Perl doesn't treat these methods any differently -- the underscore means nothing significant to Perl but is purely for human consumption. Think of it as a

\noindent 'keep out' sign, to mark the method as: for use by authorized personnel only!

\noindent 

\noindent Try It Out : Private Methods

\noindent Typically, the constructor is one place where we'll want to do a private set-up, so let's convert the code for adding to the @Everyone array into a private method:

\noindent 

\noindent package Person;

\noindent \# Class for storing data about a person

\noindent \#person8.pm

\noindent use warnings;

\noindent use strict;

\noindent use Carp;

\noindent 

\noindent my @Everyone;

\noindent 

\noindent \# Constructor and initialisation sub new \{

\noindent my \$class = shift;

\noindent 

\noindent 

\noindent my \$self = \{@\_\};

\noindent bless(\$self, \$class);

\noindent \$self-$>$\_init;

\noindent return \$self;

\noindent \}

\noindent 

\noindent sub \_init \{

\noindent my \$self = shift;

\noindent push @Everyone, \$self;

\noindent carp "New object created";

\noindent \}

\noindent 

\noindent \# Object accessor methods

\noindent sub address  \{ \$\_[0]-$>$\{address \}=\$\_[1] if defined \$\_[1]; \$\_[0]-$>$\{address \} \}

\noindent sub surname  \{ \$\_[0]-$>$\{surname \}=\$\_[1] if defined \$\_[1]; \$\_[0]-$>$\{surname \} \} sub forename \{ \$\_[0]-$>$\{forename\}=\$\_[1] if defined \$\_[1]; \$\_[0]-$>$\{forename\} \} sub phone\_no\{ \$\_[0]-$>$\{phone\_no\}=\$\_[1] if defined \$\_[1]; \$\_[0]-$>$\{phone\_no\} \} sub occupation  \{

\noindent \$\_[0]-$>$\{occupation\}=\$\_[1] if defined \$\_[1]; \$\_[0]-$>$\{occupation\}

\noindent \}

\noindent 

\noindent \# Class accessor methods

\noindent sub headcount \{ scalar @Everyone \}

\noindent sub everyone  \{ @Everyone \}

\noindent 

\[1;\] 


\noindent \textit{Try It Out}

\noindent What we've got now is pretty much the standard constructor, let's go over it again:

\noindent 

\noindent 

\noindent sub new \{

\noindent 

\noindent First, we retrieve our class name, which will be passed to us automatically when we do Class-$>$new, by using shift as a shorthand for shift @\_

\noindent 

\noindent 

\noindent my \$class = shift;

\noindent 

\noindent Then we put the rest of the arguments, which should be a hash with which to initialize the attributes, into a new hash reference:

\noindent 

\noindent my \$self = \{@\_\};

\noindent 

\noindent Now we bless the reference to tell it which class it belongs in, making it an object:

\noindent 

\noindent 

\noindent bless(\$self, \$class);

\noindent 

\noindent Do any further initialization we need to do by calling the object's private \_init method. Note that due to inheritance, this private method may be provided by a subclass.

\noindent 

\noindent \$self-$>$\_init;

\noindent 

\noindent 

\noindent Finally, return the constructed object:

\noindent 

\noindent 

\noindent return \$self;

\noindent \}

\noindent 

\noindent \textit{Utility Methods}

\noindent Our methods have mainly been accessors so far, but that's by no means all we can do with objects.

\noindent Since methods are essentially subroutines, we can do almost anything we want inside them. Let's now add some methods that do things -- \textbf{utility methods}:

\noindent 

\noindent \# Class for storing data about a person

\noindent \#person9.pm use warnings; use strict; use Carp;

\noindent 

\noindent my @Everyone;

\noindent 

\noindent \# Constructor and initialisation

\noindent \#...

\noindent 

\noindent \# Object accessor methods

\noindent \#...

\noindent 

\noindent \# Class accessor methods

\noindent \#...

\noindent 

\noindent \# Utility methods

\noindent sub fullname \{

\noindent my \$self = shift;

\noindent return \$self-$>$forename." ".\$self-$>$surname;

\noindent \}

\noindent 

\noindent sub printletter \{

\noindent my \$self = shift;

\noindent my \$name = \$self-$>$fullname;

\noindent my \$address = \$self-$>$address;

\noindent my \$forename= \$self-$>$forename;

\noindent my \$body = shift;

\noindent my @date = (localtime)[3,4,5];

\noindent \$date[1]++; \# Months start at 0! Add one to humanise!

\noindent \$date[2]+=1900;  \# Add 1900 to get current year.

\noindent my \$date = join "/", @date;

\noindent 

\noindent print $<$$<$EOF;

\noindent \$name

\noindent \$address

\noindent 

\noindent \$date

\noindent 

\noindent Dear \$forename,

\noindent 

\noindent \$body

\noindent 

\noindent Yours faithfully,

\noindent EOF

\noindent return \$self;

\noindent \}

\noindent 

\[1;\] 


\noindent 

\noindent This creates two methods, fullname and printletter. fullname returns the full name of the

\noindent person the object describes. printletter prints out a letter with a body supplied by the user. Notice that to print the name in the text of the letter, printletter itself calls fullname. It's good practice

\noindent for utility methods to return the object if they have nothing else to return. This allows you to string together calls by using the returned object as the object for the next method call, like this:

\noindent \$object-$>$one()-$>$two()-$>$three();

\noindent 

\noindent Here's an example of those utility methods in use.

\noindent 

\noindent 

\noindent \#!/usr/bin/perl

\noindent \# utility1.plx use warnings; use strict;

\noindent use Person9;

\noindent 

\noindent my \$object = Person-$>$new ( surname =$>$ "Galilei", forename =$>$ "Galileo",

\noindent address =$>$ "9.81 Pisa Apts.",

\noindent occupation =$>$ "bombadier"

\noindent );

\noindent \$object-$>$printletter("You owe me money. Please pay it.");

\noindent 

\noindent This produces our friendly demand:

\noindent 

\noindent $>$ perl utility1.plx

\noindent Galileo Galilei

\noindent 9.81 Pisa Apts.

\noindent 

\[4/5/2000\] 


\noindent Dear Galileo,

\noindent You owe me money. Please pay it. Yours faithfully,

\noindent $>$

\noindent 

\noindent \textit{Death of an Object}

\noindent We've seen how we construct an object, and we've made ourselves a constructor method that returns a

\noindent blessed reference. What happens at the end of the story, when an object needs to be destructed? Object destruction happens in two possible cases, either implicitly or explicitly:

\noindent 

\noindent ? Explicit destruction happens when no references to the object's data remains. Just like when dealing with ordinary references, you may have more than one reference to the data in existence. As we saw in Chapter 7, some of these references may be lexical variables, which

\noindent go out of scope. As they do, the reference count of the data is decreased. Once it falls to zero, the data is removed from the system.

\noindent 

\noindent ? Implicit destruction happens at the end of your program. At that point, all the data in your program is released.

\noindent 

\noindent 

\noindent When Perl needs to release data and destroy an object, whether implicitly or explicitly, it calls the

\noindent method DESTROY on the object. Unlike other utility methods, this doesn't mean Perl is telling you what

\noindent to do. Perl will destroy the data for you, but this is your chance to clean up anything else you have used, close any files you opened, shut down any network sockets, and so on. (Larry Wall joked that it should have been called something like YOU\_ARE\_ABOUT\_TO\_BE\_SHOT\_DO\_YOU\_HAVE\_ANY\_LAST\_REQUESTS instead.)

\noindent 

\noindent If Perl doesn't find a method called DESTROY, it won't complain but will silently release the object's

\noindent data. If you do provide a DESTROY method, be sure that it doesn't end up creating any more references

\noindent to the data, because that's really naughty.

\noindent 

\noindent \textit{Our Finished Class}

\noindent Let's put all the pieces of our class together finally and examine the class all the way through:

\noindent 

\noindent package Person;

\noindent 

\noindent First of all,  let me  reiterate that  a  class  is  nothing  more  than  a  package.  We  start  off our class  by

\noindent starting a  new  package.  As usual,  we  want  to  make  sure  this  package  is  at  least  as  pedantic  as  the  one that called  it,  so  we turn on  warnings  and  strictness,  and  we  load  the Carp module  to  report  errors

\noindent from the caller's perspective.

\noindent 

\noindent \# Class for storing data about a person use warnings;

\noindent use strict;

\noindent use Carp;

\noindent 

\noindent Next we declare our class attributes. These are ordinary package variables, and there's nothing special about them:

\noindent 

\noindent \# Class attributes my @Everyone;

\noindent 

\noindent We provide a nice and general constructor, which calls a private method to do its private initialization. We take the class name, create a reference, and bless it.

\noindent 

\noindent \# Constructor and initialisation sub new \{

\noindent my \$class = shift;

\noindent my \$self = \{@\_\};

\noindent bless(\$self, \$class);

\noindent \$self-$>$\_init;

\noindent return \$self;

\noindent \}

\noindent 

\noindent Our private method just adds a copy of the current object to a general pool. In more elaborate classes, we'd want to check that the user's input makes sense and get it into the format we want, open any external files we need, and so on.

\noindent 

\noindent sub \_init \{

\noindent my \$self = shift;

\noindent push @Everyone, \$self;

\noindent \}

\noindent 

\noindent 

\noindent Next we provide very simple object accessor methods to allow us to get at the keys of the hash reference

\noindent where our data is stored. These are the only interface we provide to the data inside the object, and everything goes through them.

\noindent 

\noindent \# Object accessor methods

\noindent sub address  \{ \$\_[0]-$>$\{address \}=\$\_[1] if defined \$\_[1]; \$\_[0]-$>$\{address \} \}

\noindent sub surname  \{ \$\_[0]-$>$\{surname \}=\$\_[1] if defined \$\_[1]; \$\_[0]-$>$\{surname \} \} sub forename \{ \$\_[0]-$>$\{forename\}=\$\_[1] if defined \$\_[1]; \$\_[0]-$>$\{forename\} \} sub phone\_no\{ \$\_[0]-$>$\{phone\_no\}=\$\_[1] if defined \$\_[1]; \$\_[0]-$>$\{phone\_no\} \} sub occupation  \{

\noindent \$\_[0]-$>$\{occupation\}=\$\_[1] if defined \$\_[1]; \$\_[0]-$>$\{occupation\}

\noindent \}

\noindent 

\noindent Accessing class attributes is even easier, since these are simple variables:

\noindent 

\noindent \# Class accessor methods

\noindent sub headcount \{ scalar @Everyone \}

\noindent sub everyone  \{ @Everyone \}

\noindent 

\noindent Finally, we have a couple of utility methods, which perform actions on the data in the object. The fullname method uses accessors to get at the forename and surname stored in the object and returns a string with them separated by a space:

\noindent 

\noindent \# Utility methods sub fullname \{

\noindent my \$self = shift;

\noindent return \$self-$>$forename." ".\$self-$>$surname;

\noindent \}

\noindent 

\noindent Secondly, printletter is a slightly more elaborate method that prints out a letter to the referenced person. It uses the address and forename accessors, plus the fullname method to get the object's details. Notice that in both methods we're using my \$self = shift to grab the object as it was passed to us.

\noindent 

\noindent sub printletter \{

\noindent my \$self = shift;

\noindent my \$name = \$self-$>$fullname; my \$address = \$self-$>$address; my \$forename= \$self-$>$forename; my \$body = shift;

\noindent my @date = (localtime)[3,4,5];

\noindent \$date[1]++; \# Months start at 0! Add one to humanise!

\noindent \$date[2]+=1900;  \# Add 1900 to get current year. my \$date = join "/", @date;

\noindent 

\noindent print $<$$<$EOF;

\noindent \$name

\noindent \$address

\noindent 

\noindent \$date

\noindent 

\noindent Dear \$forename,

\noindent 

\noindent \$body

\noindent 

\noindent Yours faithfully, EOF

\noindent \}

\noindent 

\[1;\] 


\noindent 

\noindent Inheritance

\noindent 

\noindent As we've gone along, we've taken certain steps to ensure that our class is suitable for having other classes inherit from it -- but what exactly do we mean by this?

\noindent 

\noindent Inheritance, like the use of modules, is an efficient way of reusing code. If we want to build an Employee class, similar to the Person class but with additional attributes (employer, position, salary) and additional methods, (hire, fire, raise, promote and so on) inheritance means we don't have to write out all the code again. Instead, we simply specify the differences.

\noindent 

\noindent How does it work? It's simple-- we tell perl the names of other classes to look in if it can't find a method.

\noindent So, in our Employee class, all we need to write are the accessors for employer, position, salary, and the new methods. We then say 'in all other respects, we're like the Person class'. We don't need to write our constructor -- when Employee-$>$new is called, perl doesn't find a subroutine called new in

\noindent our class, so it looks in Person, where, sure enough, it finds one. The same goes for all the other

\noindent methods and accessors.

\noindent 

\noindent We can also inherit from multiple sources. To do this, we give a list of classes to look in. Perl will

\noindent consult each class in order, using the first method it finds. We can also have a chain of inheritance -- we can create an Employee::Programmer derived from the Employee class but with a fixed position

\noindent and with additional methods (\$geek-$>$read("userfriendly"), \$geek-$>$drink("cola"),

\noindent \$geek-$>$hack("naked"), and so on).

\noindent 

\noindent \textit{What Is It?}

\noindent The fact that we can give a \textit{list }of classes should give you an idea of how we do this -- we use an array. Specifically, the package global @ISA is used to tell Perl what our class is derived from. I must admit

\noindent that for a long time I pronounced it I-S-A, which didn't help me understand it one bit. If you haven't got

\noindent it yet, read it as two words: 'is a'. For example, @Employee::ISA = ("Person"), or we could say

\noindent @Employee::Programmer::ISA = ("Employee"), and grammar be damned.

\noindent 

\noindent Try It Out : Inheriting from a class

\noindent First of all, let's create a subclass that is exactly the same as Person. The 'empty subclass test' ensures that we can inherit from a class. Create a file Employee1.pm, and put the following in it:

\noindent 

\noindent package Employee;

\noindent \#Employee1.pm use Person9; use warnings; use strict;

\noindent 

\noindent our @ISA = qw(Person);

\noindent 

\noindent That's all we need to do to create a new class based on Person. We now have at our disposal all the methods that Person provides, and we can test this by changing our examples to use Employee instead of Person:

\noindent 

\noindent \#!/usr/bin/perl

\noindent \# inherit1.plx use warnings; use strict;

\noindent use Employee1;

\noindent 

\noindent 

\noindent my \$object = Employee-$>$new (

\noindent surname =$>$ "Galilei", forename =$>$ "Galileo",

\noindent address =$>$ "9.81 Pisa Apts.", occupation =$>$ "bombadier"

\noindent );

\noindent 

\noindent \$object-$>$printletter("You owe me money. Please pay it.");

\noindent 

\noindent This does exactly the same as before.

\noindent 

\noindent \textit{How It Works}

\noindent This is how our new class is constructed:

\noindent 

\noindent package Employee;

\noindent 

\noindent We provide a package declaration to start the class. Now, if we're going to bring in the Person class, we'd better make sure that the file that contains it is loaded. We ensure this by loading it:

\noindent 

\noindent use Person9;

\noindent 

\noindent Good habits dictate that we include these lines:

\noindent 

\noindent use warnings;

\noindent use strict;

\noindent 

\noindent Finally, we come to the action -- the package array @ISA, which does the magic:

\noindent 

\noindent our @ISA = qw(Person);

\noindent 

\noindent Perl looked for Employee::new. But since we didn't specify one, it used Person::new instead. The same goes for Employee::printletter. With this one line, we've reproduced the entire class, or rather, Perl has done it all for us, behind the scenes. Nice and easy.

\noindent 

\noindent Next we need to extend the class to provide our new methods.

\noindent 

\noindent \textit{Adding New Methods}

\noindent At this stage, adding new methods is easy -- just define them:

\noindent 

\noindent 

\noindent package Employee;

\noindent \#Employee2.pm

\noindent use Person9;

\noindent use warnings;

\noindent use strict;

\noindent 

\noindent our @ISA = qw(Person);

\noindent 

\noindent sub employer \{ \$\_[0]-$>$\{employer\}=\$\_[1] if defined \$\_[1]; \$\_[0]-$>$\{employer\} \}

\noindent sub position \{ \$\_[0]-$>$\{position\}=\$\_[1] if defined \$\_[1]; \$\_[0]-$>$\{position\} \}

\noindent sub salary \{ \$\_[0]-$>$\{salary  \}=\$\_[1] if defined \$\_[1]; \$\_[0]-$>$\{salary  \} \}

\noindent 

\noindent 

\noindent sub raise \{

\noindent my \$self = shift;

\noindent my \$newsalary = \$self-$>$salary + 2000;

\noindent \$self-$>$salary(\$newsalary);

\noindent return \$self;

\noindent \}

\noindent 

\noindent Now we can add and change these additional attributes:

\noindent 

\noindent 

\noindent \#!/usr/bin/perl

\noindent \# inherit2.plx use warnings; use strict;

\noindent use Employee2;

\noindent 

\noindent my \$object = Employee-$>$new ( surname =$>$ "Galilei", forename =$>$ "Galileo",

\noindent address =$>$ "9.81 Pisa Apts.",

\noindent occupation =$>$ "bombadier"

\noindent );

\noindent 

\noindent \$object-$>$salary("12000");

\noindent print "Initial salary: ", \$object-$>$salary, "\textbackslash n";

\noindent print "Salary after raise: ", \$object-$>$raise-$>$salary, "\textbackslash n";

\noindent 

\noindent $>$\textbf{perl inherit2.plx}

\noindent Initial salary: 12000

\noindent Salary after raise: 14000

\noindent $>$

\noindent 

\noindent \textit{Overriding Methods}

\noindent As well as adding new methods, we can provide our own version of the old ones. We certainly should provide our own version of \_init, since that's a private method. In this case, we'll replace the

\noindent employer with another Person object:

\noindent 

\noindent 

\noindent sub \_init \{

\noindent my \$self = shift;

\noindent my \$employer = \$self-$>$employer \textbar \textbar  "unknown";

\noindent unless (ref \$employer) \{

\noindent my \$new\_o = Person-$>$new( surname =$>$ \$employer );

\noindent \$self-$>$employer(\$new\_o);

\noindent \}

\noindent \$self-$>$SUPER::\_init();

\noindent \}

\noindent 

\noindent Now when we create a new Employee object, the constructor will call \$self-$>$\_init. This will now

\noindent be found in our class, and the subroutine above will be run. What does it do?

\noindent 

\noindent sub \_init \{

\noindent 

\noindent 

\noindent As usual, we get the object we were passed:

\noindent 

\noindent my \$self = shift;

\noindent 

\noindent From the object, we use the employer accessor we provided to extract the employer data from the

\noindent object. If the user didn't provide an employer when calling the constructor, we use the word 'unknown':

\noindent 

\noindent my \$employer = \$self-$>$employer \textbar \textbar  "unknown";

\noindent 

\noindent Now, the employer may already be an object, if so, we don't need to do anything else:

\noindent 

\noindent unless (ref \$employer) \{

\noindent 

\noindent Otherwise, we create a new object, and assign that as the current employee's employer:

\noindent 

\noindent my \$new\_o = Person-$>$new( surname =$>$ \$employer );

\noindent \$self-$>$employer(\$new\_o);

\noindent \}

\noindent 

\noindent Now for this wonderfully cryptic piece:

\noindent 

\noindent \$self-$>$SUPER::\_init();

\noindent 

\noindent What on earth is SUPER::? Well, SUPER::method means 'call this method in the superclass'. In our case, once we've finished doing what we need to do, we tell Perl to call Person::\_init so that the Person class can have a chance to arrange things as it likes.

\noindent 

\noindent Here's our complete employee class:

\noindent 

\noindent package Employee;

\noindent \#Employee3.pm

\noindent use Person9;

\noindent use warnings;

\noindent use strict;

\noindent our @ISA = qw(Person);

\noindent sub employer \{ \$\_[0]-$>$\{employer\}=\$\_[1] if defined \$\_[1]; \$\_[0]-$>$\{employer\} \}

\noindent sub position \{ \$\_[0]-$>$\{position\}=\$\_[1] if defined \$\_[1]; \$\_[0]-$>$\{position\} \}

\noindent sub salary \{ \$\_[0]-$>$\{salary  \}=\$\_[1] if defined \$\_[1]; \$\_[0]-$>$\{salary  \} \}

\noindent 

\noindent sub raise \{

\noindent my \$self = shift;

\noindent my \$newsalary = \$self-$>$salary + 2000;

\noindent \$self-$>$salary(\$newsalary);

\noindent return \$self;

\noindent \}

\noindent 

\noindent sub \_init \{

\noindent my \$self = shift;

\noindent my \$employer = \$self-$>$employer \textbar \textbar  "unknown";

\noindent unless (ref \$employer) \{

\noindent my \$new\_o = Person-$>$new( surname =$>$ \$employer );

\noindent \$self-$>$employer(\$new\_o);

\noindent \}

\noindent \$self-$>$SUPER::\_init();

\noindent \}

\noindent 

\noindent 

\noindent Now let's see what we can do with it. We start off our new program as usual:

\noindent 

\noindent 

\noindent \#!/usr/bin/perl

\noindent \# inherit3.plx

\noindent use warnings;

\noindent use strict;

\noindent use Employee3;

\noindent 

\noindent We create a new Employee object, and a new employer:

\noindent 

\noindent 

\noindent my \$dilbert = Employee-$>$new (

\noindent surname =$>$ "Dilbert",

\noindent employer =$>$ "Dogbert",

\noindent salary =$>$ "43000"

\noindent );

\noindent 

\noindent This automatically creates a new Person as the employer, so we can now get at Dogbert and change \textit{his}

\noindent attributes:

\noindent 

\noindent 

\noindent my \$boss = \$dilbert-$>$employer;

\noindent \$boss-$>$address("3724 Cubeville");

\noindent 

\noindent Of course, there's nothing to say that the employer has to be a Person object. There are always bigger fish:

\noindent 

\noindent 

\noindent my \$dogbert = Employee-$>$new (

\noindent surname =$>$ "Dogbert",

\noindent employer =$>$ "PHB",

\noindent salary =$>$ \$dilbert-$>$salary*2

\noindent );

\noindent \$dilbert-$>$employer(\$dogbert);

\noindent 

\noindent This creates a new employee object for Dogbert, with his boss recorded as 'the PHB'. We put this object

\noindent in Dilbert's employer data. We can now get at the PHB in two ways:

\noindent 

\noindent 

\noindent my \$phb = \$dogbert-$>$employer;

\noindent 

\noindent Or, starting at the bottom of the chain, since \$dilbert-$>$employer is \$dogbert:

\noindent 

\noindent 

\noindent my \$phb = \$dilbert-$>$employer-$>$employer;

\noindent 

\noindent 

\noindent \textit{There's one class of which everything is a subclass, the UNIVERSAL class, which is is the prime mover in Perl's OO world. You'll hardly ever use it, but it will provide you with some methods. isa(\$package) will return true if your class inherits from that package. can(\$method) is true if your class can perform the named method, and VERSION returns the value of the package variable \$VERSION in your class, if one exists.}

\noindent 

\noindent 

\noindent Ties

\noindent 

\noindent As we've already mentioned, one particular thing that Perl gives us is the ability to hide an object behind a simple variable. What this means is that if we access such a variable, perl will call an object method behind our back; we can use this to do all sorts of magical things with our variables. This is called 'tying', and we say that the variable is \textbf{tied }to the class. When we looked at

\noindent Win32::TieRegistry, the registry, a file on the disk, was accessed through an ordinary-looking hash,

\noindent \%Registry. To tie a variable to a class, we use the tie statement, like this:

\noindent 

\noindent tie \$variable, 'Class', @parameters;

\noindent 

\noindent This you can tie any kind of variable -- a scalar, an array, a hash, and even a filehandle. tie returns the object that we use to manipulate the variable, but we don't usually take much notice of that -- the idea of tie is to keep the object hidden away from the program.

\noindent 

\noindent There's nothing special we need to do to our class to tell Perl we're going to tie to it. However, if we do

\noindent tie a variable, perl expects to be able to call certain object and class methods. So, for instance, when you

\noindent tie a scalar as above, perl will call the TIESCALAR method in the class Class. In fact, it'll do this:

\noindent 

\noindent \$tied = Class-$>$TIESCALAR(@parameters);

\noindent 

\noindent This should be an ordinary constructor, setting up whatever it is we're trying to do with this variable

\noindent and returning an object. If the variable you're tying is an array, perl expects to call TIEARRAY. We also have TIEHASH and TIEHANDLE for hashes and filehandles, respectively. You don't need to provide all four constructors, just the one for the type of variable your class should be tied to. As a nice simple example, we'll create a 'counter', which increases in value every time you access it:

\noindent 

\noindent package Autoincrement;

\noindent \#autoincrement.pm

\noindent use warnings;

\noindent use strict;

\noindent 

\noindent sub TIESCALAR \{

\noindent my \$class = shift; \# No parameters

\noindent my \$realdata = 0;

\noindent return bless \textbackslash \$realdata, \$class;

\noindent \}

\noindent 

\[1;\] 


\noindent We'll keep the real value of the counter in a scalar, and we'll have our object being a blessed reference

\noindent to that scalar. We can now tie a scalar to this class:

\noindent 

\noindent \#!/usr/bin/perl

\noindent \# tiescalar.plx

\noindent use warnings;

\noindent use strict;

\noindent use Autoincrement;

\noindent 

\noindent my \$count;

\noindent tie \$count, 'Autoincrement';

\noindent 

\noindent 

\noindent The next thing we need to do is define what happens when we access the variable: Perl will call various

\noindent object methods depending on what we do. For example, when we try and retrieve the value of the variable, by saying print \$count or similar, Perl will try and call the FETCH method. Try it now -- put this at the end of the program:

\noindent 

\noindent 

\noindent print \$count;

\noindent 

\noindent You should see the following error:

\noindent 

\noindent Can't locate object method "FETCH" via package "Autoincrement" at tiescalar.plx line 9.

\noindent 

\noindent So, we need to provide a method called FETCH. What should it do? Well, in our case, we need to look inside the object reference and get out the real value of our counter, and we need to increase it by one. We can do these both at the same time, by adding this method:

\noindent 

\noindent 

\noindent sub FETCH \{

\noindent my \$self = shift;

\noindent return \$\$self++;

\noindent \}

\noindent 

\noindent Notice that the value we return from our method is what we want our tied variable, \$count, to produce. Now let's print it a few more times and see what happens:

\noindent 

\noindent 

\noindent \#!/usr/bin/perl

\noindent \# tiescalar.plx use warnings; use strict;

\noindent use Autoincrement;

\noindent 

\noindent my \$count;

\noindent tie \$count, 'Autoincrement';

\noindent print \$count, "\textbackslash n";

\noindent print \$count, "\textbackslash n";

\noindent print \$count, "\textbackslash n";

\noindent print \$count, "\textbackslash n";

\noindent 

\noindent You'll see the following:

\noindent 

\noindent $>$\textbf{perl tiescalar.plx}

\[0\] 

\[1\] 

\[2\] 

\[3\] 
$>$

\noindent 

\noindent Our FETCH method allows us to dictate what \$count will produce, programmatically; in our case, it's producing a number that will increment each time we read it.

\noindent 

\noindent What else can we do with a scalar? Well, as well as fetch data from it, we can store data to it. The relevant method we need to provide for that is STORE. With an auto-incrementing variable, we'll fix things so that every attempt to store something to it winds us back to zero:

\noindent 

\noindent 

\noindent sub STORE \{

\noindent my \$self = shift;

\noindent \$\$self = 0;

\noindent \}

\noindent 

\noindent Note that when we try to store something to \$count now, it doesn't destroy the fact that it's a tied variable; the only way to remove the special treatment it gets is to untie the variable, like this:

\noindent 

\noindent 

\noindent untie \$count;

\noindent 

\noindent So now we can write to our variable, and it'll just reset the counter to zero:

\noindent 

\noindent 

\noindent \#!/usr/bin/perl

\noindent \# tiescalar2.plx

\noindent use warnings;

\noindent use strict;

\noindent use Autoincrement;

\noindent 

\noindent my \$count;

\noindent tie \$count, 'Autoincrement';

\noindent print \$count, "\textbackslash n";

\noindent print \$count, "\textbackslash n";

\noindent print \$count, "\textbackslash n";

\noindent print \$count, "\textbackslash n";

\noindent \$count = "Bye bye!";

\noindent print \$count, "\textbackslash n";

\noindent print \$count, "\textbackslash n";

\noindent print \$count, "\textbackslash n";

\noindent print \$count, "\textbackslash n";

\noindent 

\noindent $>$\textbf{perl tiescalar2.plx}

\[0\] 

\[1\] 

\[2\] 

\[3\] 

\[0\] 

\[1\] 

\[2\] 

\[3\] 
$>$

\noindent 

\noindent In this case, we didn't do anything with the value the program tried to store (we ignored the "Bye

\noindent Bye!"). However, if we want to get hold of the value, it's passed as a parameter to our method:

\noindent 

\noindent 

\noindent sub STORE \{

\noindent my \$self  = shift;

\noindent my \$value = shift;

\noindent warn "Hi, you said \$value\textbackslash n";

\noindent \$\$self = 0;

\noindent \}

\noindent 

\noindent 

\noindent will now cause our program to print:

\noindent 

\noindent $>$\textbf{perl tiescalar2.plx}

\[0\] 

\[1\] 

\[2\] 

\[3\] 
Hi, you said Bye, bye!

\[0\] 

\[1\] 

\[2\] 

\[3\] 
$>$

\noindent 

\noindent This is all we need to do to control the treatment of a scalar, because it's all a scalar can do -- initialize itself, retrieve a scalar value, and store a scalar value. Arrays, hashes, and filehandles are trickier, but let's quickly recap what we've provided:

\noindent 

\noindent TIESCALAR(@parameters) Construct an object ready to be used for a tied scalar.

\noindent 

\noindent FETCH() Retrieve the value of a tied scalar, however we wish to do so.

\noindent 

\noindent STORE(\$value) Actions to perform when \$value is stored into our tied scalar.

\noindent 

\noindent 

\noindent For an array, there are some extra methods you need to provide:

\noindent 

\noindent TIEARRAY(@parameters) Construct an object ready to be used for a tied scalar.

\noindent 

\noindent FETCH(\$element) Fetch the \$element'th element of the array.

\noindent 

\noindent STORE(\$element, \$value) Store \$value in the \$element'th element of the array.

\noindent 

\noindent FETCHSIZE() Provide the size of the array, for when the user says

\noindent \$\#tied\_array or scalar @tied\_array.

\noindent 

\noindent STORESIZE(\$size) Store the size, for when the user says

\noindent \$\#tied\_array=\$size or similar.

\noindent 

\noindent 

\noindent If you're implementing tied arrays, it's recommended that you inherit from the class Tie::StdArray

\noindent that lives in the standard Tie::Array package. Tie::StdArray provides you with a tied equivalent

\noindent for a standard Perl array, and you should overload whichever methods you need to customize it for your purposes.

\noindent 

\noindent Likewise, ties can be inherited from the Tie::StdHash package in Tie::Hash. This will expect your underlying object to be a blessed hash, and provide EXISTS and DELETE which are called by the Perl operators of the same name, as well as the key iterators FIRSTKEY and NEXTKEY which are used when the user calls keys or each on the tied hash. These iterators are not things you want to write yourself,

\noindent and it's far easier to inherit from Perl if possible.

\noindent 

\noindent To close, here's a very, very simple way of making a hash persistent that is, the data in the hash will be stored on disk when your program finishes so you can use it again next time.

\noindent 

\noindent 

\noindent Try It Out : Simple Persistent Tied Hashes

\noindent We'll store a hash on disk in the following format:

\noindent 

\noindent 

\noindent Key:value

\noindent Key:value

\noindent ...

\noindent 

\noindent We know how to read this in -- we'll just split each line on a colon, like this:

\noindent 

\noindent 

\noindent if ( -e \$file ) \{

\noindent open FH, \$file or die \$!;

\noindent while ($<$FH$>$) \{

\noindent chomp;

\noindent my (\$k, \$v) = split /:/,\$\_,2;

\noindent \$realhash\{\$k\} = \$v;

\noindent \}

\noindent \}

\noindent 

\noindent Let's turn this into our constructor:

\noindent 

\noindent 

\noindent package PersistHash;

\noindent \#persisthash.pm

\noindent use strict;

\noindent use warnings;

\noindent use Tie::Hash;

\noindent our @ISA = qw(Tie::StdHash);

\noindent sub TIEHASH \{

\noindent my \$class = shift;

\noindent my \%realhash;

\noindent my \$file = shift;

\noindent if ( -e \$file ) \{

\noindent open FH, \$file or die \$!;

\noindent while ($<$FH$>$) \{

\noindent chomp;

\noindent my (\$k, \$v) = split /:/,\$\_,2;

\noindent \$realhash\{\$k\} = \$v;

\noindent \}

\noindent \} \# Otherwise we'll create it when we're done.

\noindent \$realhash\{\underbar{  }secret\underbar{  }\} = \$file; \# Need to stash this for when we write.

\noindent return bless \textbackslash \%realhash, \$class;

\noindent \}

\noindent 

\noindent Now we have something that operates just like a standard Perl hash, with the exception that we read the

\noindent data from a file when the hash is initially tied. We can say:

\noindent 

\noindent 

\noindent tie \%dictionary, "PersistHash", "mydict.txt";

\noindent 

\noindent and the data will be loaded from the file. So far we can manipulate and change this data -- but we've no way of writing it back when we're finished. The hash can go out of scope, become untied, or reach the

\noindent end of the program, and we won't have written its contents to disk.

\noindent 

\noindent 

\noindent In each of these cases though, the destructor will be called to clear up the object -- and that's exactly

\noindent when we need to save the data back to the disk. This is why we have destructors!

\noindent 

\noindent 

\noindent sub DESTROY \{

\noindent my \$self = shift;

\noindent my \%realhash = \%\$self;

\noindent my \$file = \$realhash\{\underbar{  }secret\underbar{  }\}; \# Extract the filename we stashed.

\noindent delete \$realhash\{\underbar{  }secret\underbar{  }\}; \# Don't want that written to disk.

\noindent open FH, "$>$\$file" or die \$!;

\noindent for (keys \%realhash) \{

\noindent print FH \$\_, ":", \$realhash\{\$\_\}, "\textbackslash n";

\noindent \}

\noindent close FH;

\noindent \}

\noindent 

\noindent There we are -- we can load the data in, we can write it back out again, and with the exception of

\noindent providing the file name, this all takes places without the user knowing they're dealing with an object. We've provided what looks like a normal hash, but we've hidden away the persistence.

\noindent 

\noindent Of course, this is a very simplified implementation; reading in a line at a time means we'll have

\noindent problems if the keys or values the user wants to store contain new lines or if they contain colons. Storing references and objects in the hash won't work at all, so it's impossible to hold deep data structures.

\noindent Worst of all, we store the filename inside the hash itself, which feels like a messy solution; it's unlikely that anyone would want to get at the value \_\_secret\_\_, but unlikely things do happen, particularly in programming.

\noindent 

\noindent Ideally, we'd separate that sort of data from the hash, and we'd also ensure that the data would get written in such a way that neither newlines nor colons, nor indeed anything we choose to store in our hash, will break the storage. Thankfully, this has already been done, and we'll look at the module that implements it in the next chapter.

\noindent 

\noindent 

\noindent Summary

\noindent 

\noindent Object-oriented programming is another way of thinking about programming. You approach it in terms

\noindent of data and the relationships between pieces of data, which we call objects. These objects belong to divisions called classes -- these have properties (\textbf{attributes}) and can perform activities (\textbf{methods}).

\noindent 

\noindent Perl makes object-oriented programming neat and simple:

\noindent 

\noindent ? An \textbf{object }is a reference that has been blessed into a class.

\noindent 

\noindent ? A \textbf{class }is an ordinary Perl package.

\noindent 

\noindent ? A \textbf{method }is an ordinary Perl subroutine.

\noindent 

\noindent From these three basic principles, we can start to build data-driven applications. We've seen how easy it

\noindent is to apply \textbf{inheritance}, deriving a more specific class (a \textbf{subclass}) from a more general class (a

\noindent \textbf{superclass}) by merely specifying what's different.

\noindent 

\noindent 

\noindent 

\noindent 

\noindent Exercises

\noindent 

\noindent 1.   Give Dogbert a phone number, a position, and a value for anything else that's undefined.

\noindent Make sure that you can see exactly what is happening in the code by printing out values at each stage in inherit3.plx.

\noindent 

\noindent 2.   Create a new object method, to print out a business card. It must print out all the pertinent information of the chosen employee.

\noindent  

\noindent  

\noindent  

\noindent  

\noindent 

\noindent 

\noindent 

\noindent This work is licensed under the Creative Commons Attribution-NoDerivs-NonCommercial License. To view a copy of this

\noindent license, visit http://creativecommons.org/licenses/by-nd-nc/1.0 or send a letter to Creative Commons, 559 Nathan Abbott Way, Stanford, California 94305, USA.

\noindent 

\noindent The key terms of this license are:

\noindent 

\noindent Attribution: The licensor permits others to copy, distribute, display, and perform the work. In return, licensees must give the original author credit.

\noindent 

\noindent No  Derivative  Works: The licensor permits others to copy, distribute, display and perform only unaltered copies of the work -- not derivative works based on it.

\noindent 

\noindent Noncommercial: The licensor permits others to copy, distribute, display, and perform the work. In return, licensees may not use the work for commercial purposes -- unless they get the licensor's permission.


\end{document}

